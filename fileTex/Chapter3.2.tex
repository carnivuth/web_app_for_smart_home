\section{YouCook: Scenari}

\FloatBarrier
\begin{longtable}{|p{4.5cm}|p{10cm}|}
\hline
\endfirsthead
\multicolumn{2}{c}%
{\tablename\ \thetable\ -- \textit{Continued from previous page}} \\
\hline
\endhead
\hline \multicolumn{2}{r}{\textit{Continued on next page}} \\
\endfoot
\hline
\endlastfoot


         \textbf{Titolo} & Login \\

         \hline
         \textbf{Descrizione} & Permette di accedere al sistema \\

         \hline
         \textbf{Attori} & Utente\\

         \hline
         \textbf{Relazioni} & Amministrazione gruppo, Amministrazione ricette \\
         \hline
         \textbf{Precondizioni} & L'utente deve essere già registrato\\

         \hline
         \textbf{Postcondizioni} & L'utente è autenticato nel sistema\\

         \hline
         \textbf{Scenario principale} & 
            \begin{enumerate}
                \item All'utente viene presentata una maschera per l'inserimento delle credenziali
                \item L'utente inserisce i propri dati
                \item Il sistema verifica che i dati inseriti siano corretti
                \item Viene presentata la schermata principale all'utente 
            \end{enumerate}
            \\


         \hline
         \textbf{Scenari alternativi} & Scenario a: Credenziali non riconosciute
            \begin{enumerate}
                \setcounter{enumi}{3}
                \item Il sistema non riconosce le credenziali inserite e presenta nuovamente la schermata iniziale di accesso
            \end{enumerate}
         \\

         \hline
         \textbf{Requisiti non funzionali} &\\

         \hline
         \textbf{Punti aperti} & \\



\end{longtable}
\FloatBarrier
\begin{longtable}{|p{4.5cm}|p{10cm}|}
\hline
\endfirsthead
\multicolumn{2}{c}%
{\tablename\ \thetable\ -- \textit{Continued from previous page}} \\
\hline
\endhead
\hline \multicolumn{2}{r}{\textit{Continued on next page}} \\
\endfoot
\hline
\endlastfoot


         \textbf{Titolo} & Amministrazione ricette \\

         \hline
         \textbf{Descrizione} & Permette di eliminare, modificare e aggiungere ricette  \\

         \hline
         \textbf{Attori} & Utente\\

         \hline
         \textbf{Relazioni} & creazione ricetta, modifica ricetta, elimina ricetta, login \\
         \hline
         \textbf{Precondizioni} & L'utente deve essere già autenticato\\

         \hline
         \textbf{Postcondizioni} &L'utente ha modificato il suo elenco di ricette\\

         \hline
         \textbf{Scenario principale} & 
            \begin{enumerate}
                \item All'utente viene presentata una maschera con all'interno la lista delle ricette presenti nel suo inventario 
                \item L'utente seleziona l'opzione che vuole effettuare (modifica, aggiunta o eliminazione)
                \item Il sistema porta a compimento l'azione descritta dall'utente
            \end{enumerate}
            \\


         \hline
         
         \hline
         \textbf{Requisiti non funzionali} &\\

         \hline
         \textbf{Punti aperti} & \\



\end{longtable}
\FloatBarrier
\begin{longtable}{|p{4.5cm}|p{10cm}|}
\hline
\endfirsthead
\multicolumn{2}{c}%
{\tablename\ \thetable\ -- \textit{Continued from previous page}} \\
\hline
\endhead
\hline \multicolumn{2}{r}{\textit{Continued on next page}} \\
\endfoot
\hline
\endlastfoot


         \textbf{Titolo} & Amministrazione Gruppo famiglia \\

         \hline
         \textbf{Descrizione} & Permette di eliminare e aggiungere mebri al gruppo famiglia  \\

         \hline
         \textbf{Attori} & Utente\\

         \hline
         \textbf{Relazioni} & aggiunta membro, rimozione membro, login \\
         \hline
         \textbf{Precondizioni} & L'utente deve essere già autenticato\\

         \hline
         \textbf{Postcondizioni} &L'utente ha modificato il suo gruppo famiglia\\

         \hline
         \textbf{Scenario principale} & 
            \begin{enumerate}
                \item All'utente viene presentata una maschera con all'interno la lista dei mebri del suo gruppo famiglia 
                \item L'utente seleziona l'opzione che vuole effettuare (aggiunta o eliminazione)
                \item Il sistema porta a compimento l'azione descritta dall'utente
            \end{enumerate}
            \\


         \hline
         
         \hline
         \textbf{Requisiti non funzionali} &\\

         \hline
         \textbf{Punti aperti} & \\



\end{longtable}