\section{Routing dei componenti Angular}
Una delle funzionalità fondamentali per una single page application è la possibilità di navigare tra le varie viste senza interrogare il server per reperire una nuova pagina.
La libreria Angular consente di navigare tra le viste del' applicazione tramite il package routing, distribuito esternamente rispetto al core.
\newline
\newline
La libreria implementa la funzionalità tramite l'oggetto routing, gestito tramite il pattern singleton e creato all'avvio dell'applicazione, l'oggetto è in grado di tradurre gli url in componenti da renderizzare all'interno della pagina.
Le associazioni tra url e componenti vengono effettuate all'interno della definizione del ngModule, tramite il metodo 
\begin{minted}{javascript}
    RouterModule.forRoot();
\end{minted}
al quale viene fornito, come argomento, un array contenete le coppie url componente.
La renderizzazione dell'output del routing angular avviene all'interno della direttiva
\begin{minted}{html}
    <router-outlet></router-outlet>
\end{minted}
Alla fine di ogni ciclo di navigazione, la libreria costruisce un albero per rappresentare lo stato corrente del router, disponibile a tutti i componenti tramite il service Router. 
\newpage