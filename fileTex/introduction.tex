\addcontentsline{toc}{chapter}{Introduzione}
\chapter*{Introduzione}
In questo elaborato verranno discusse le metodologie con cui ottimizzare le prestazioni del front end di applicazioni web.
In particolare, verranno discusse le motivazioni per cui è importante farlo. Verrà discusso lo stato attuale delle applicazioni nel web con particolare riguardo per l'ottimizzazione dei tempi di caricamento delle pagine.
\newline
L'obbiettivo del' elaborato è offrire un punto di vista, sullo sviluppo delle applicazioni web, che sia più inclusivo delle questioni e problematiche riguardanti le performance che, nello sviluppo web moderno, non è molto presente e viene spesso e erroneamente messo in secondo piano, quando in realtà influenza la user experience in maniera rilevante e può determinare da solo l'abbandono dell'applicativo da parte dell'utente.
\newline
Fra i tanti framework per lo sviluppo web verrà mostrato Angular. È stato scelto questo framework per la sua capacità di personalizzazione del comportamento a runtime tramite la change detection e il ciclo di vita dei componenti e, come verrà mostrato, per la sua natura rivolta alle performance.
\newline
Angular, inoltre offre allo sviluppatore un insieme di concetti completo e di alto livello, adatto allo sviluppo sia di applicazioni frontend di facciata come possono essere siti prevalentemente statici, sia a single page application data intensive che necessitano di alte performance. Inoltre angular offre un comodo ambiente di sviluppo, fornito di tool di corredo per la generazione del codice e per il tesing runtime dei suoi componenti.
\newline
Verranno analizzati i principali concetti che il framework mette a disposizione agli sviluppatori fra cui la programmazione delle interfacce orientata ai componenti,il data binding a due vie, la loro gestione a runtime, la capacità organizzativa offerta dalla libreria tramite i moduli e le modalità di implementazione della dependency injection tramite il concetto dei service.
\newline
Verrà posta particolare attenzione sul meccanismo di aggiornamento del DOM che la libreria propone,chiamato change detection.
La change detection è il meccanismo che il framework utilizza per il monitoraggio e aggiornamento della view.
Verrà mostrato il suo funzionamento, le modalità con cui la libreria lo implementa, le possibilità di ottimizzazione tramite la strategia onPush e le accortezze da prendere in fase di sviluppo per un corretto utilizzo dello stesso.
\newline
A fini dimostrativi, verrà incluso il progetto di un applicazione per la gestione elettronica delle ricette di cucina che metta in mostra le capacità del framework, con particolare riguardo verso il meccanismo di change detection. Verranno mostrati i requisiti, la struttura dell'applicazione e la sua progettazione con test conclusivi delle prestazioni.
L'applicazione ha come obbiettivo quello di soppiantare il classico libro di ricette tenuto in casa realizzandone una versione elettronica, offrendo inoltre la possibilità di creare gruppi famiglia all'interno del quale condividere le ricette


