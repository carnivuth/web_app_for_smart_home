\addcontentsline{toc}{chapter}{Conclusioni}
\chapter*{Conclusioni}

Come mostrato nell'elaborato, le prestazioni in termini di tempo di risposta delle applicazioni sono cruciali quando si vuole fornire una valida user experience. L'utente moderno del web vuole applicazioni responsive e non desidera perdere tempo dietro a schermate di caricamento. In un contesto simile l'ottimizzazione delle prestazioni deve essere un punto cruciale dello sviluppo delle applicazioni web sin dalle prime fasi del progetto.
\newline
In questo contesto la libreria Angular si è evoluta molto dalla sua prima creazione nel 2016. Essa offre una suite di concetti completa e all'avanguardia, in grado di soddisfare le necessità di sviluppo di applicazioni complesse e data intensive senza sacrificare le prestazioni.
\newline
Il concetto di componente proposto dalla libreria suddivide la visualizzazione dei dati e la logica applicativa dalla rappresentazione a schermo in maniera ottimale, consente di fare dialogare i due mondi agilmente tramite il data binding, sollevando lo sviluppatore da questo compito e consente di limitare la progettazione del componente solo a questi due aspetti, delegando altri compiti di utilità a concetti diversi come i service.
\newline 
La libreria non si risparmia nell'implementazione del concetto di modularità e tramite i module è possibile realizzare applicazioni composte da elementi assestanti e in grado di essere importati in altre applicazioni e utilizzati in maniera autonoma.
\newline
La dependency injection fornita dall'applicazione tramite il concetto di service rende i componenti ignari delle logiche di utilità necessarie all'applicazione per funzionare, questo consente di pensare i componenti come compartimenti stagni che, tramite i service, si interfacciano con il resto dell'applicazione.
\newline 
La capacita di influenzare il ciclo di vita dei componenti e il meccanismo di change detection consentono di effettuare un preciso tuning delle prestazioni e della gestione degli eventi a runtime dell'applicazione, questo offre allo sviluppatore un ambiente runtime molto performante e con una alta possibilità di personalizzazione per adattarsi alle esigenze della applicazione. Il meccanismo di change detection può essere letteralmente modellato "addosso all'applicazione" definendo, per ogni componente, come l'ambiente a runtime deve comportarsi per eseguire l'aggiornamento dell'interfaccia.
\newline 
La libreria, nell'ottica di ridurre al minimo il tempo speso per operazioni di corredo, offre inoltre un vasto ambiente per la generazione del codice e per il testing delle applicazioni.
\newline 
L'applicazione YouCook mostrata nell'elaborato e sviluppata sfruttando le funzionalità offerte dal framework, consente di amministrare la propria libreria di ricette fornendo una versione elettronica del tradizionale ricettario di carta.
Grazie ai concetti offerti dal framework la progettazione risulta facilmente estendibile e modulare consentendo di variare il datastore e di estendere le interfacce tramite l'aggiunta di componenti e service.
\newline 
Per queste motivazioni la libreria Angular si dimostra una delle migliori soluzioni per lo sviluppo di applicazioni web ad alte prestazioni.

